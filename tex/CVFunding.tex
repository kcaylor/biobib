% CV Funding Table
% Created on 2023-06-29 17:47

\begin{longtable}{p{1.75cm}>{\raggedright}p{2.cm}p{6cm}}
Years & Agency/Source & Title & Total Amount & Total to UCSB & Personal Share & Role & New/Cont.\\
\hline 
\endfirsthead

\multicolumn{3}{c}%
{{Funding - continued from previous page }} \\ \\
Years & Agency/Source & Title & Total Amount & Total to UCSB & Personal Share & Role & New/Cont.\\
\hline 
\endhead

\\
\multicolumn{3}{c}%
{{ Funding continued on next page }} \\
\endfoot

\hline \hline
\endlastfoot

\multicolumn{1}{l} {\bf Total :} & \$23,711,262 & & \\

2023-2028 & Department of Defense & The Climate-Food-Urbanization Nexus and the Precursors of Instability in Africa, Co-PI (\$3,073,150) \\ 
2023-2025 & Department of Energy & Improving ESS approaches to evapotranspiration partitioning through data fusion, Co-PI (\$399,007) \\ 
2023-2023 & National Science Foundation & NSF Convergence Accelerator Track J: Predicting the effect of climate extremes on the food system to improve resilience of global and local food security, Co-PI (\$748,674) \\ 
2022-2025 & Zegar Foundation & Ecohydrological Resilience of Vegetation to Water Availability and Drought, PI (\$672,079) \\ 
2019-2023 & National Science Foundation & CNH2-L:  Linkages and Interactions Between Urban Food Security and Rural Agricultural Systems, Co-PI (\$1,600,000) \\ 
2019-2021 & Zegar Foundation & Developing and Implementing Novel Tools for Assessing Riparian Ecosystem Resilience at the Dangermond Preserve, PI (\$321,583) \\ 
2018-2022 & Department of Defense, SERDP & Strategic Environmental Research \& Development: Understanding and Assessing Riparian Habitat Vulnerability to Drought-Prone Climate Regimes on Department of Defense Bases in the Southwestern US 
, Co-PI (\$1,704,236) \\ 
2018-2019 & National Geographic Society & Global maps of center pivot agriculture for improving estimates of land use change and water use, PI (\$100,811) \\ 
2018-2018 & The Nature Conservancy & Task 25: Dangermond Preserve Bren Summer Internships and Group Project, PI (\$17,000) \\ 
2018-2019 & National Science Foundation & WSC-Category 2 Collaborative: Impacts of Agricultural Decision Making and Adaptive Management on Food Security, PI (\$628,779) \\ 
2018-2019 & Omidyar Foundation & Developing and Scaling up the Mapping Africa Active Learning Platform, PI (\$79,838) \\ 
2017-2019 & Clark University & Hazards SEES: Understanding Cross-Scale Interactions of Trade and Food Policy to Improve Resilience to Drought Risk, PI (\$103,249) \\ 
2017-2020 & National Science Foundation & Collaborative Research: Impacts of Dynamic Climate-Driven Water Availability on Tree Water Use and Health in Mediterranean Riparian Forests 
, Co-PI (\$396,566) \\ 
2017-2020 & National Science Foundation & CC*Networking Infrastructure: UCSB Network Upgrade to 100 Gigabit. , PI (\$481,730) \\ 
2015-2019 & National Science Foundation & Hazards SEES: Understanding Cross-Scale Interactions of Trade and Food Policy to Improve Resilience to Drought Risk, Co-PI (\$2,519,689) \\ 
2015-2015 & National Science Foundation & I-Corps: Pulsepod: Bringing the Cloud Down to Earth, PI (\$50,000) \\ 
2014-2018 & National Science Foundation & WSC-Category 2 Collaborative: Impacts of Agricultural Decision Making and Adaptive Management on Food Security, PI (\$1,929,565) \\ 
2013-2014 & JPL & Adding precision to agricultural prac- tice through simultaneous chlorophyll fluorescence and infrared observations, Co-PI (\$100,000) \\ 
2013-2014 & JPL & Rapid forest triage by sub-canopy mi- cro air vehicle, Co-PI (\$100,000) \\ 
2013-2015 & Princeton University & Drought and the global carbon cycle, Co-PI (\$200,000) \\ 
2012-2016 & National Science Foundation & Collaborative research: Quantifying the ecosystem-wide impacts of a strong interactor in African watersheds, PI (\$123,336) \\ 
2011-2014 & NASA & Multiscale effects of fire on long-term climate and precipitation, Co-I (\$1,149,128) \\ 
2011-2016 & National Science Foundation & CNH: Institutional Dynamics of Adaptation to Climate Change: Longitudinal Analysis of Snowmelt-Dependent Agricultural Systems, Co-PI (\$1,200,000) \\ 
2011-2014 & Princeton University & A global collaborative network for coupling hydrological forecasts and food security in Sub-Saharan Africa and China , PI (\$225,000) \\ 
2010-2011 & Princeton University & The Importance of Green Feedbacks on Biodiversity Conservation and Management of Savanna Ecosystems, PI (\$100,000) \\ 
2010-2015 & National Science Foundation & ESE: Collaborative Research: Spatial Resilience of Agriculturalists to Coupled Ecological and Hydrological Variability in Rural Zambia, PI (\$350,000) \\ 
2009-2014 & National Science Foundation & CAREER: An ecohydrological framework for understanding land degradation in dryland ecosystems, PI (\$1,543,738) \\ 
2009-2011 & National Science Foundation & US - Kenya Doctoral Dissertation Enhancement Project: The Impact of Macropores on the Spatial and Temporal Patterns of Soil Moisture in Dryland Ecosystems of Central Kenya, PI (\$15,000) \\ 
2009-2009 & Li-COR Biosciences & Using the LI-6400XT in the Princeton Semester in Kenya Program: From Undergraduate Education to International Impact, PI (\$45,000) \\ 
2008-2013 & National Science Foundation & LTREB: KLEE- scaling up and out at the Kenya Long-term Exclosure Experiment, Co-PI (\$449,800) \\ 
2007-2010 & National Science Foundation & Collaborative Proposal: Distribution and Dynamics of Belowground Carbon in Savannas, PI (\$208,344) \\ 
2007-2008 & Indiana University & Quantifying organic and pharmaceutical contaminants within urbanizing streams in a central Indiana urbanizing watershed, PI (\$51,323) \\ 
2007-2010 & National Science Foundation & HSD: Dynamics of Reforestation in Coupled Social-Ecological Systems: Modeling Land-Use Decision Making and Policy Impacts, Co-PI (\$755,115) \\ 
2007-2008 & Monroe County Drainage Board & Quantifying the impacts of development on physical and chemical hydrology of central Indiana streams , PI (\$96,000) \\ 
2006-2010 & National Science Foundation & IRES: US-South Africa International Research Experience for Students -- Impacts of Disturbance on Above- and Below-ground Structure of Southern African Savannas, Co-PI (\$623,522) \\ 
2005-2007 & NASA & Sources, Transports, and Impacts of Southern African Aerosols: Synthetic Case Studies Using Terra \& Aqua Satellite Products, Co-I (\$650,000) \\ 
2005-2007 & NASA & Hydrologic \& Nutrient Controls on the Structure and Function of Southern African Savannas: A Multi-Scale Approach

, Co-I (\$900,000) \\ 

\end{longtable}

